\documentclass{scrbook}

\begin{document}
%\chapter{Introduction}
\chapter{Deployment}
\section{Prerequisites}
The framework does not require any particular operating system.
We will assume that you want to install Scavenger 2.x on a Linux cluster
using only the shell (e.g. bash).

You do not need any administrator rights on the system where you want
to install the Scavenger framework. 
All you need is a working Java Runtime Environment (> 1.6) on the cluster, 
and some way to connect to the cluster (presumably \lstinline{ssh} and \lstinline{scp}).

You will need the following resources and software to be able to install and run the
Scavenger 2.x framework:

\begin{itemize}
  \item A working JRE installation (at least 1.6, later versions should work fine too)
  \item Internet access, browser + \lstinline{scp}, \lstinline{git} or at least \lstinline{wget} to download
    the necessary JARs.
\end{itemize}

\subsection{Adding plugins to SBT}
TODO: sbt-pack plugin

\section{Obtaining the software}
\section{Trying it out locally}
\section{Installation on a cluster}
\section{Updating the Scavenger 2.x framework}
Currently, the Scavenger 2.x framework does not provide any built-in mechanism for
software updates. If you want to get a never version, delete the TODO directory,
download and unpack the newer version, and adjust the paths in your \lstinline{.bashrc}.

\section{Configuration}
\section{Running a Scavenger 2.x application}
\section{Running with LSF}
%\chapter{Writing applications for the Scavenger 2.x framework}
\end{document}
