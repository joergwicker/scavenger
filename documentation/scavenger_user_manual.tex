\documentclass{scrbook}

\usepackage{courier}
\usepackage{listings}
\lstset{basicstyle=\footnotesize\ttfamily,breaklines=true}
\lstdefinelanguage{scala}{
  morekeywords={abstract,case,catch,class,def,%
    do,else,extends,false,final,finally,%
    for,if,implicit,import,match,mixin,%
    new,null,object,override,package,%
    private,protected,requires,return,sealed,%
    super,this,throw,trait,true,try,%
    type,val,var,while,with,yield},
  otherkeywords={=>,<-,<\%,<:,>:,\#,@},
  sensitive=true,
  morecomment=[l]{//},
  morecomment=[n]{/*}{*/},
  morestring=[b]",
  morestring=[b]',
  morestring=[b]"""
}
\usepackage{helvet}
\renewcommand{\familydefault}{\sfdefault}
\usepackage{geometry}
\geometry{bindingoffset=1cm} % Make pages appear more or less centered

\usepackage{hyperref}
\usepackage{xcolor}
\hypersetup{
    colorlinks,
    linkcolor={blue!80!black},
    urlcolor={blue!50!black}
}


\begin{document}
%\chapter{Introduction}
\chapter{Deployment}
\section{Prerequisites}
The framework itself does not rely on any particular operating system,
but some of the scripts that come together with the framework are
currently available only for Linux and Unix. 
Porting them to Windows is possible, but not entirely trivial.
We will assume that you want to install a Scavenger 2.x application 
on a Linux cluster using only the shell (e.g. bash).
We will use different prompts to distinguish between shells on
the cluster and on your local machine:
\begin{lstlisting}
  cluster currentDirectory> someCommandIssuedOnTheCluster
  local currentDirectory> someCommandIssuedOnYourLocalMachine
\end{lstlisting}
We will omit \lstinline{currentDirectory} 
if it's not important or can be inferred
from the context.

You do not need any administrator rights on the system where you want
to install a Scavenger 2.x application.
Essentially, all you need is a working Java Runtime Environment 
(at least version 1.6) on the cluster, and some way to connect to the cluster 
(presumably \lstinline{ssh} and \lstinline{scp}). Everything else can be
obtained and installed without administrator rights.

Here is the list of resources, tools and changes in configuration
that you will need in order to build, install and run Scavenger 2.x applications. 
We first give the full list, and later describe some of the items in more detail.

\noindent\textbf{On the cluster (installed by administrator)}
\begin{itemize}
  \item A working JRE installation 
    (at least 1.6, later versions should work fine too)
  \item Access with \lstinline{ssh} and \lstinline{scp}
  \item Optionally: LSF or some other job-scheduling system
\end{itemize}

\noindent\textbf{On the cluster (uploaded or configured by you)}
\begin{itemize}
  \item Bash scripts from \lstinline{scavenger_2_x/bin/*}
  \item Added environment variable \lstinline{SCAVENGER_HOME} and modified 
    \lstinline{PATH}
  \item Per application: a modified \lstinline{scavenger.conf} file 
  \item Per application: directory with all the necessary jars
\end{itemize}

\noindent\textbf{On the machine where you build your project}
\begin{itemize}
  \item A working JRE installation 
  \item SBT (Scala Build Tool)
  \item \lstinline{sbt-pack} plug-in
\end{itemize}

Notice that you can use Maven or any other build tool of your choice, however,
in the following we will assume that you stick with SBT.

\subsection{Obtaining SBT and installing plugins}
Visit the SBT-website \url{http://www.scala-sbt.org/0.13/tutorial/Setup.html} and install SBT on your local machine following the instructions.
Then open the file 
\begin{lstlisting}
  ~/.sbt/0.13/plugins/plugin.sbt
\end{lstlisting} 
and append the following line to the end:
\begin{lstlisting}
  addSbtPlugin("org.xerial.sbt" % "sbt-pack" % "0.7.5")
\end{lstlisting}
This will tell the SBT that the \lstinline{sbt-pack} plugin is 
required. The actual installation is handled automatically by SBT.

\subsection{Obtaining Scavenger 2.x scripts and JARs}
Currently, there are no Maven/SBT-repositories containing Scavenger 2.x.
You can either download and unzip or git-clone the Scavenger 2.x project from 
\url{https://github.com/joergwicker/scavenger} \footnote{
  A less stable ``bleeding-edge''-version can be found here:
  \url{https://github.com/tyukiand/scavenger_2_x}
}.
The result should be a directory (we will assume that it's called \lstinline{scavenger_2_x}) with the content that 
looks approximately as follows:
\begin{lstlisting}
  local scavenger_2_x> ls
  bin
  build.sbt
  documentation
  scavenger.conf
  src
\end{lstlisting}
Here is a brief description of what all these files and directories are:
\begin{description}
  \item[bin] Contains bash-scripts that substantially simplify the
    starting of Scavenger 2.x nodes
  \item[build.sbt] Description of SBT-build, necessary to build the 
    Scavenger 2.x jars
  \item[documentation] Source code of this documentation
  \item[scavenger.conf] A basic version of the Scavenger 2.x configuration
    file. You will have to modify it in order to be able to start your
    Scavenger 2.x application on multiple nodes.
  \item[src] Source code of the Scavenger 2.x framework.
\end{description}

Now you have the SBT build tool and all necessary files to build 
the Scavenger 2.x framework. From the directory containing the
\lstinline{build.sbt} file, issue the following command in your shell:
\begin{lstlisting}
  local scavenger_2_x> sbt publishLocal
\end{lstlisting}
This will compile Scavenger 2.x, and package it into a jar-file 
(you will be able to find it in the newly created subdirectory
called \lstinline{target}). 
Then, sbt will publish it locally by creating bunch of \lstinline{.pom} and \lstinline{.jar}-files in the directory 
\begin{lstlisting}
~/.ivy2/local/org.scavenger/scavenger_<x>/<y>
\end{lstlisting} 
Now these JARs can be 
used by other projects that have Scavenger 2.x as a dependency.

\subsection{Making bash-scripts available}
As mentioned previously, the framework comes together with bash-scripts
that facilitate launching of Scavenger 2.x nodes. These scripts can 
be found in \lstinline{scavenger_2_x/bin}. In order to be able to use them, we have to copy them on the cluster, and then adjust some environment variables. First, copy the entire \lstinline{scavenger_2_x} directory (with compiled and packaged \lstinline{target}) to the cluster. For Mogon, it could look
as follows:
\begin{lstlisting}
  local> scp -r scavenger_2_x \
    username@mogon.zdv.uni-mainz.de:/home/username/path/to/
\end{lstlisting}
where \lstinline{/path/to/} stands for some path in the cluster
file system.

Now, edit your \lstinline{~/.bashrc} configuration file (on the cluster), and add the following two lines:
\begin{lstlisting}
  export SCAVENGER_HOME=/path/to/scavenger_2_x
  export PATH=$PATH:$SCAVENGER_HOME/bin
\end{lstlisting}
Save the file and restart bash (or simply \lstinline{source} the
modified \lstinline{.bashrc}).
You should be able to see something when issuing the command 
\begin{lstlisting}
  cluster> echo $SCAVENGER_HOME
\end{lstlisting}
The script called \lstinline{scavenger} should also become visible.
Try it out by issuing the command
\begin{lstlisting}
  cluster> scavenger --help
\end{lstlisting}
It should print an overview of available commands and options.

Moreover, now we already should be able to start a seed node and
worker nodes (even though they can't do anything useful so far).
To test this, issue the following command:

\begin{lstlisting}
  cluster> scavenger startSeed \
    --config-file 'path/to/scavenger_2_x/scavenger.conf'
\end{lstlisting}
It will print a wall of (mostly incomprehensible) messages and hang
forever. Create a second bash-instance, and issue the command
\begin{lstlisting}
  cluster> scavenger startWorker \
    --port 2553 \
    --config-file 'path/to/scavenger_2_x/scavenger.conf'
\end{lstlisting}
Again, it should print a wall of messages. What one should
observe is that the launching of the worker node triggers some
activity in the console with the seed node. If one looks closely,
one might be able to see some messages that indicate that the 
connection between the seed node and the worker node has been 
established. Terminate both seed and worker forcibly by 
\lstinline{Ctrl+C} and close the second shell.

If you want to be able to test your application locally, 
repeat the modification of environment variables on your local 
machine.

\section{Building the application}
Let us summarize what we have done so far:
\begin{itemize}
  \item We have built and published Scavenger 2.x locally, so 
    that we can use it as dependency in other SBT and Maven 
    projects.
  \item We have copied JARs and bash scripts to the cluster,
    and adjusted the environment variables on the 
    cluster, so that we can now start seed and worker nodes
    using the \lstinline{scavenger} script.
\end{itemize}
We want to make use of the first item from the above list, 
and create a new SBT project that has Scavenger 2.x 
as dependency. We will demonstrate the building process on
a simple piece of code.
You can either follow the instructions step by step, or 
clone the full example from the repository
\url{https://github.com/tyukiand/scavenger_2_x_demo}.

If you want to build the example from scratch, 
begin by creating an empty directory, let's call it 
\lstinline{myScavengerDemo}.
In this directory, create a file called \lstinline{build.sbt}
with the following content:
\begin{lstlisting}[language=scala]
lazy val root = (project in file(".")).
  settings(
    // choosing name and version for the example is up to you
    organization := "org.foobar", 
    name := "ScavengerHelloWorld",
    version := "3.14", 
    // this should be the same as the Akka-version used by scavenger.
    scalaVersion := "2.10.4", 
    libraryDependencies ++= Seq(
      // this must be the same as the locally published version
      "org.scavenger" % "scavenger_2.10" % "2.1" 
    ),
    packAutoSettings
  ) // don't forget any parentheses
\end{lstlisting}
This is an SBT build description. Here, we give some information about the
project (name, version, organization), specify a scala version (you can 
use a newer one, but make sure it's compatible with the Akka version used
by your current Scavenger 2.x installation, check the dependencies listed
in \lstinline{scavenger_2_x/build.sbt}). Then we say that our project 
depends on the Scavenger 2.1 framework. The exact version numbers are not
that important: they can (and will) change. What is important is that 
the specified version numbers are the same as in your current
\lstinline{scavenger_2_x/build.sbt}, otherwise SBT will not be able 
to find the dependency.

Now, create a source-code directory within the \lstinline{myScavengerDemo}
directory:
\begin{lstlisting}
  local myScavengerDemo> mkdir -p src/main/scala/scavenger_demo
\end{lstlisting}
In this directory, create a file called \lstinline{DistributedHelloWorld.scala}
with the following content:

\begin{lstlisting}[language=scala]
package scavenger_demo
import scala.concurrent.Future
import scavenger._
import scavenger.app.DistributedScavengerApp
import scala.concurrent.Future

case object Rot13 extends AtomicAlgorithm[Char, Char] {
  def apply(c: Char, ctx: Context): Future[Char] = {
    // just sleep a few milliseconds, pretend it's hard
    Thread.sleep((new scala.util.Random).nextInt(1000)) 
    val res = 
      if ('A' <= c && c <= 'Z') ('A' + (c + 13 - 'A') % 26).toChar
      else if ('a' <= c && c <= 'z') ('a' + (c + 13 - 'a') % 26).toChar
      else c
    Future(res)(ctx.executionContext)
  }
  def identifier = scavenger.categories.formalccc.Atom("rot13")
  def difficulty = Expensive
}

/**
 * Scavenger 2.x-specific hello-world example,
 * workers are in separate JVM's
 */
object DistributedHelloWorld extends DistributedScavengerApp {
  def main(args: Array[String]): Unit = {

    // before we do anything, we have to initialize the master node
    scavengerInit()

    // Here is some data to work with.
    val encryptedMessage = "Uryyb, Jbeyq!"

    // Each character becomes a trivial computation
    val data = for (character <- encryptedMessage) yield {
      Computation("" + character, character)
    }

    // Now we apply `rot13` algorithm to the data and 
    // generate bunch of jobs: one job for each character
    val jobs = for (trivialComp <- data) yield {
      Rot13(trivialComp)
    }

    // Nothing has been computed so far. We have merely created
    // descriptions of jobs. Here is the full list of jobs that
    // we want to submit:
    println("Full list of jobs:")
    jobs foreach println


    // submit all jobs, `Computation[Char]` are 
    // transformed into `Future[Char]`
    val singleCharFutures = for (j <- jobs) yield {
      scavengerContext.submit(j)
    }
 
    // collect all futures into a single one, 
    // concatenate single chars into a message
    val resultFuture = 
      for (seq <-  Future.sequence(singleCharFutures)) yield {
        seq.mkString("")
      }
    
    // install a callback on the result: 
    // print it and shutdown as soon as
    // the decrypted message is there
    resultFuture.onSuccess { case str: String =>
      // draw the result with a box around it
      println("#" * 80)
      println("#" + (" " * 78) + "#")
      println("#" + (" " * ((78 - str.size) / 2)) + str + 
        (" " * (78 - (78 - str.size) / 2 - str.size)) + "#") 
      println("#" + (" " * 78) + "#")
      println("#" * 80)
      scavengerShutdown()
    }

  }
}
\end{lstlisting}
This program takes a ROT13-encrypted ``Hello, World!'' message, transforms 
each character into a separate job, decrypts each character on a different
worker node, puts the partial results together, and prints out the decrypted
message. 
The implementation details are not important right now, because we want to 
demonstrate how to package and to deploy the application.

Now you can compile the code as follows (the current directory should be the
one that contains \lstinline{build.sbt} of your example):
\begin{lstlisting}
  local myScavengerDemo> sbt compile
\end{lstlisting}
In order to launch the application, we need one more ingredient: 
the application-specific configuration file.

\section{Configuration}
Each Scavenger 2.x application requires a configuration file. 
This configuration file contains the address and port of the seed node, and
also some settings relevant for the master node and the worker nodes.
The directory \lstinline{scavenger_2_x} contains a basic version of such a 
configuration file, it's called \lstinline{scavenger.conf}. 
Copy this file into the directory \lstinline{myScavengerDemo}. 
It is a plain-text file with a format based on JSON. 
Open it with the text-editor of your choice. The most important variables 
are 
\begin{lstlisting}
scavenger {
  seed {
    akka.remote.netty.tcp.hostname = "login02"
    akka.remote.netty.tcp.port = 9876
  }
}
\end{lstlisting}
These two variables must contain the right address and the right port such that
the worker nodes and the master node can find it. 
For example, on Mogon, it is reasonable to start the seed on the login node. 
In such a case, we one should use something like
\lstinline{"login01"} or \lstinline{"login02"}
as the host name (depending on where you are starting the
seed node). One should also choose the port number 
in such a way that it doesn't cause any conflicts with 
other applications (we will use 9876 as example).
Notice that the host name is a string (double quotes), 
while port number is a numeric value (no quotes).

\section{Deploying applications on the cluster}
So far, we have written the source-code of our application, compiled it, 
and supplied a configuration file. However, the compiled classes and the 
configuration file are not sufficient to run the application on the cluster,
because we also need the JARs with all the dependencies.

In the \lstinline{build.sbt}, we have added the setting
\lstinline{packAutoSettings}. 
This configures the \lstinline{sbt-pack} plugin.
This plugin helps us to collect all the required jars in a single directory,
so that they can be easily uploaded to the cluster. If you are using Maven,
you can take a look at the \lstinline{dependencies} plugin, specifically at
the \lstinline{copy-dependencies} command.

Package and collect all transitive dependencies into a single 
directory by issuing the following command:
\begin{lstlisting}
  local myScavengerDemo> sbt pack
\end{lstlisting}
This will generate a directory \lstinline{target/pack/lib} with a whole lot of
JARs (this is why we don't want to collect all the
dependencies manually). Now you can copy all these jars to the cluster:
\begin{lstlisting}
  local myScavengerDemo> scp -r target/pack/lib \
    username@cluster:/path/to/the/jars
\end{lstlisting}
You should also copy the modified configuration file
\lstinline{scavenger.conf} to the cluster:
\begin{lstlisting}
  local myScavengerDemo> scp -r scavenger.conf \
    username@cluster:/path/to/the/confFile
\end{lstlisting}

\section{Running the application}
Now all the JARs and the modified configuration file are on the cluster, 
and all relevant Bash scripts are available, 
so we should be able to launch our application.

Begin with the seed-node (remember that it has to be launched 
on the node specified in the configuration file, otherwise
all the other nodes won't be able to find it):
\begin{lstlisting}
  cluster> scavenger startSeed \
    --port 9876
    --config-file path/to/modified/scavenger.conf
\end{lstlisting}

For a moment assume that you have direct access to other
nodes of your cluster (this is not the case if you have to
use LSF, we will discuss this use-case in the next section).

Login to another node, and start a worker node as follows:
\begin{lstlisting}
  cluster> scavenger startWorker \
    --jars 'path/to/uploaded/jars'
    --port 9876
\end{lstlisting}
Notice that all the worker nodes must have access to the
JARs that your application depend on, otherwise they won't
be able to deserialize incoming job-messages.

Finally, launch the master from some other node.
The master node runs together with the actual client
application, the name of the client application must 
be specified explicitly:
\begin{lstlisting}
  cluster> scavenger startMaster \
    --jars 'target/pack/lib/*' \
    --port 9876 \
    --main scavenger_demo.DistributedHelloWorld
\end{lstlisting}

Now the worker node should be able to connect to the
seed node, the master node should also be able to connect 
to the seed node. Then the master will establish a connection
to the worker node, and switch into normal operation mode.
Then it should decrypt the \lstinline{"Hello, World!"} message
character by character, print it out, and exit.

\section{Running with LSF}
On Mogon, you will have no ssh-access to other physical 
nodes. Instead, you must save the above commands into 
scripts, \emph{specify ton of weird options (TODO...)},
and submit the scripts using \lstinline{bsub}.

\section{Updating the Scavenger 2.x framework}
Currently, the Scavenger 2.x framework does not provide any built-in mechanism for
software updates. If you want to get a never version, either
git-pull a newer version and re-build everything, or simply
delete the entire \lstinline{scavenger_2_x} directory that you have cloned/downloaded previously, and install everything again.

%\chapter{Writing applications for the Scavenger 2.x framework}
\end{document}
